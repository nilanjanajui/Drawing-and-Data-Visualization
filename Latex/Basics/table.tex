\documentclass{article}
\usepackage[T1]{fontenc}
\usepackage{array}
\usepackage{booktabs}

\begin{document}
\begin{tabular}{lll}
  Animal & Food  & Size   \\
  dog    & meat  & medium \\
  horse  & hay   & large  \\
  frog   & flies & small  \\
\end{tabular}

\newpage

\begin{tabular}{cl}
  Animal & Description \\
  dog    & The dog is a member of the genus Canis, which forms part of the
           wolf-like canids, and is the most widely abundant terrestrial
           carnivore. \\
  cat    & The cat is a domestic species of small carnivorous mammal. It is the
           only domesticated species in the family Felidae and is often referred
           to as the domestic cat to distinguish it from the wild members of the
           family. \\
\end{tabular}


% The issue is that the l type column typesets its contents in a single row at its natural width, even if there is a page border in the way. To overcome this you can use the p column. This typesets its contents as paragraphs with the width you specify as an argument and vertically aligns them at the top – which you’ll want most of the time. Compare the above outcome to the following:

\newpage
\begin{tabular}{cp{9cm}}

  Animal & Description \\
  dog    & The dog is a member of the genus Canis, which forms part of the
           wolf-like canids, and is the most widely abundant terrestrial
           carnivore. \\
  cat    & The cat is a domestic species of small carnivorous mammal. It is the
           only domesticated species in the family Felidae and is often referred
           to as the domestic cat to distinguish it from the wild members of the
           family. \\
\end{tabular}


\newpage

% If your table has many columns of the same type it is cumbersome to put that many column definitions in the preamble. You can make things easier by using *{num}{string}, which repeats the string num times. So *{6}{c} is equivalent to cccccc.

\begin{tabular}{*{3}{l}}
  Animal & Food  & Size   \\
  dog    & meat  & medium \\
  horse  & hay   & large  \\
  frog   & flies & small  \\
\end{tabular}


\newpage 
% adding rules(lines)
% Three of the rule commands are: \toprule, \midrule, and \bottomrule.

\begin{tabular}{lll}
  \toprule
  Animal & Food  & Size   \\
  \midrule
  dog    & meat  & medium \\
  horse  & hay   & large  \\
  frog   & flies & small  \\
  \bottomrule
\end{tabular}

\newpage

% The fourth rule command provided by booktabs is \cmidrule. It can be used to draw a rule that doesn’t span the entire width of the table but only a specified column range. A column range is entered as a number span: {number-number}.

\begin{tabular}{lll}
  \toprule
  Animal & Food  & Size   \\
  \midrule
  dog    & meat  & medium \\
  \cmidrule{1-2}
  horse  & hay   & large  \\
  \cmidrule{1-1}
  \cmidrule{3-3}
  frog   & flies & small  \\
  \bottomrule
\end{tabular}

\newpage

% Sometimes a rule would be too much of a separation for two rows but to get across the meaning more clearly you want to separate them by some means. In this case you can use \addlinespace to insert a small skip.


\begin{tabular}{cp{9cm}}
  \toprule
  Animal & Description \\
  \midrule
  dog    & The dog is a member of the genus Canis, which forms part of the
           wolf-like canids, and is the most widely abundant terrestrial
           carnivore. \\
  \addlinespace
  cat    & The cat is a domestic species of small carnivorous mammal. It is the
           only domesticated species in the family Felidae and is often referred
           to as the domestic cat to distinguish it from the wild members of the
           family. \\
  \bottomrule
\end{tabular}


\newpage

% Margin Cells

% can merge cells horizontally by using the \multicolumn command. It has to be used as the first thing in a cell. \multicolumn takes three arguments:

%The number of cells which should be merged
%The alignment of the merged cell
%The contents of the merged cell

\begin{tabular}{lll}
  \toprule
  Animal & Food  & Size   \\
  \midrule
  dog    & meat  & medium \\
  horse  & hay   & large  \\
  frog   & flies & small  \\
  fuath  & \multicolumn{2}{c}{unknown} \\
  \bottomrule
\end{tabular}


\newpage

% also use \multicolumn on a single cell to prevent the application of whatever you defined in the table preamble for the current column. The following uses this method to center the table’s head row:

\begin{tabular}{lll}
  \toprule
  \multicolumn{1}{c}{Animal} & \multicolumn{1}{c}{Food} & \multicolumn{1}{c}{Size} \\
  \midrule
  dog    & meat  & medium \\
  horse  & hay   & large  \\
  frog   & flies & small  \\
  fuath  & \multicolumn{2}{c}{unknown} \\
  \bottomrule
\end{tabular}

\newpage

% Merging cells vertically isn’t supported by LaTeX. Usually it suffices to leave cells empty to give the reader the correct idea of what was meant without explicitly making cells span rows.

\begin{tabular}{lll}
  \toprule
  Group     & Animal & Size   \\
  \midrule
  herbivore & horse  & large  \\
            & deer   & medium \\
            & rabbit & small  \\
  \addlinespace
  carnivore & dog    & medium \\
            & cat    & small  \\
            & lion   & large  \\
  \addlinespace
  omnivore  & crow   & small  \\
            & bear   & large  \\
            & pig    & medium \\
  \bottomrule
\end{tabular}


\end{document}
