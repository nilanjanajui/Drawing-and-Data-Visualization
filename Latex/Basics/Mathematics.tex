\documentclass{article}
\usepackage{amsmath}
\newcommand{\diff}{\mathop{}\!d}            % For italic
% \newcommand{\diff}{\mathop{}\!\mathrm{d}} 		% For upright
\begin{document}

% Math Mode

A sentence with inline mathematics: $y = mx + c$.
A second sentence with inline mathematics: $5^{2}=3^{2}+4^{2}$.


A second paragraph containing display math.
\[
  y = mx + c
\]
See how the paragraph continues after the display. \\

1. Superscripts $a^{b}$ and subscripts $a_{b}$ .		\\

2. Some mathematics: $y = 2 \sin \theta^{2}$.  \\

% Display Mathematics

3. A paragraph about a larger equation
\[
\int_{-\infty}^{+\infty} e^{-x^2} \, dx
\]		\\


% We’ve added one piece of manual spacing here: \, makes a thin space before the dx. Formatting of the differential operator varies: some publishers use an upright ‘d’ whilst others use an italic ‘d’. One way to write your source to allow you to handle either is to create a command \diff that you can adjust as required,

4. A paragraph about a larger equation
\[
\int_{-\infty}^{+\infty} e^{-x^2} \diff x
\]		\\

% The amsmath package

Solve the following recurrence for $ n,k\geq 0 $:
\begin{align*}
  Q_{n,0} &= 1   \quad Q_{0,k} = [k=0];  \\
  Q_{n,k} &= Q_{n-1,k}+Q_{n-1,k-1}+\binom{n}{k}, \quad\text{for $n$, $k>0$.}
\end{align*}		\\

% The align* environment makes the equations line up on the ampersands, the & symbols, just like a table. Notice how we’ve used \quad to insert a bit of space, and \text to put some normal text inside math mode. We’ve also used another math mode command, \binom, for a binomial.
%Notice that here we used align*, and the equation didn’t come out numbered. Most math environments number the equations by default, and the starred variant (with a *) disables numbering.


% Example for Matrices
AMS matrices.
\[
\begin{matrix}
a & b & c \\
d & e & f
\end{matrix}
\quad
\begin{pmatrix}
a & b & c \\
d & e & f
\end{pmatrix}
\quad
\begin{bmatrix}
a & b & c \\
d & e & f
\end{bmatrix}
\]			\\


% fonts in math mode
%Unlike normal text, font changes in math mode often convey very specific meaning. They are therefore often written explicitly. There are a set of commands you need here:

%\mathrm: roman (upright)
%\mathit: italic spaced as ‘text’
%\mathbf: boldface
%\mathsf: sans serif
%\mathtt: monospaced (typewriter)
%\mathbb: double-struck (blackboard bold) (provided by the amsfonts package)

Fonts:
The matrix $\mathbf{M}$.			\\

%Note that the default math italic separates letters so that they may be used to denote a product of variables. Use \mathit to make a word italic.

%The \math.. font commands use fonts specified for math use. Sometimes you need to embed a word that is part of the outer sentence structure and needs the current text font, for that you can use \text{...} (which is provided by the amsmath package) or specific font styles such as \textrm{..}.


$\text{bad use } size  \neq \mathit{size} \neq \mathrm{size} $

\textit{$\text{bad use } size \neq \mathit{size} \neq \mathrm{size} $}




\end{document}