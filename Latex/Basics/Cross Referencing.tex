

\documentclass{article}


\begin{document}

% The \label and \ref mechanism
Hey world!

This is a first document.

\section{Title of the first section}

Text of material for the first section.


\subsection{Subsection of the first section}
\label{subsec:labelone}

Text of material for the first subsection.
\begin{equation}
  e^{i/\pi}+1 = 0
\label{eq:labeltwo}
\end{equation}

In subsection~\ref{subsec:labelone} is equation~\ref{eq:labeltwo}.



% Where to put \label
% \texttt{The \label command always refers to the previous numbered entity: a section, an equation, a float, etc. That means that \label always has to come after the thing you want to refer to. In particular, when you create floats, the \label has to come after (or better, in), the \caption command, but within the float environment.}



\end{document}